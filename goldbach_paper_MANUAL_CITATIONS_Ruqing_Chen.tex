\documentclass[11pt]{article}

% ============================================================================
% Packages
% ============================================================================
\usepackage[utf8]{inputenc}
\usepackage[T1]{fontenc}
\usepackage{amsmath,amssymb,amsthm}
\usepackage{graphicx}
\usepackage{booktabs}
\usepackage{multirow}
\usepackage{float}
\usepackage{hyperref}
\usepackage[margin=1in]{geometry}
\usepackage{authblk}
\usepackage{xcolor}
\usepackage{algorithm}
\usepackage{algorithmic}
\usepackage{caption}
\usepackage{subcaption}

% ============================================================================
% Theorem environments
% ============================================================================
\newtheorem{theorem}{Theorem}
\newtheorem{lemma}[theorem]{Lemma}
\newtheorem{proposition}[theorem]{Proposition}
\newtheorem{corollary}[theorem]{Corollary}
\theoremstyle{definition}
\newtheorem{definition}{Definition}
\newtheorem{remark}{Remark}

% ============================================================================
% Custom commands
% ============================================================================
\newcommand{\C}{\mathbb{C}}
\newcommand{\R}{\mathbb{R}}
\newcommand{\Z}{\mathbb{Z}}
\newcommand{\N}{\mathbb{N}}
\newcommand{\Q}{\mathbb{Q}}
\DeclareMathOperator{\Li}{Li}
\DeclareMathOperator{\E}{E}

% ============================================================================
% Title and author information
% ============================================================================
\title{\textbf{Hardy-Littlewood Goldbach Conjecture Validated to $N=10^{12}$: \\
From Transient U-Distribution to Ultimate Asymptotic Convergence}}

\author{Ruqing Chen}
\affil{GUT Geoservice Inc., Montreal, Quebec, Canada\\
\texttt{ruqing@hotmail.com}}

\date{\today}

\begin{document}

\maketitle

% ============================================================================
% Abstract
% ============================================================================
\begin{abstract}
We present the first comprehensive validation of Hardy-Littlewood Goldbach formula spanning 10 orders of magnitude ($N \in [10^3, 10^{12}]$). Through strategic sampling of 126 exact counts ($N \leq 2.25 \times 10^8$), ultra-large scale validation at $N=10^9$, and Monte Carlo probing at $N=10^{12}$, we observe complete convergence of bias from $-7.87\%$ ($N=10^3$) to $+0.28\% \pm 2.2\%$ ($N=10^{12}$), marking the first documented sign reversal. At finite scales ($N<10^8$), bias exhibits U-shaped dependence on $\omega(N)$ (number of distinct prime factors), with $\omega=2$ showing maximum deviation. A convergence model $\text{Bias} \sim -58.8/\ln(N) - 17.0/\ln^2(N)$ achieves excellent fit (residual std $= 2.11\%$) and predicts bias $<0.3\%$ for all $N>10^{12}$. High-order polynomial regression achieves $R^2=0.84$ within training range but diverges significantly when extrapolated ($+12.3\%$ predicted vs. $-0.5\%$ actual at $N=10^9$), validating our convergence analysis. These findings provide unprecedented computational verification of Hardy-Littlewood asymptotics and demonstrate that U-shaped finite-scale distribution is a transient integration artifact rather than structural deficiency. At $N=10^{12}$, systematic bias diminishes to $1/10$ of statistical noise, achieving the computational ``noise floor'' where Hardy-Littlewood formula attains statistical perfection.
\end{abstract}

\textbf{Keywords:} Goldbach conjecture, Hardy-Littlewood formula, asymptotic convergence, ultra-large scale validation, transient phenomena, computational number theory

\textbf{MSC 2020:} 11P32, 11Y35, 11N05, 11Y16

% ============================================================================
% 1. Introduction
% ============================================================================
\section{Introduction}

\subsection{Background and Historical Context}

The Goldbach conjecture, proposed in 1742, asserts that every even integer $N > 2$ can be expressed as the sum of two primes. Let $G(N)$ denote the number of such representations (counting order). While the conjecture remains unproven, Hardy and Littlewood~[1] provided a heuristic asymptotic formula in their seminal 1923 paper:

\begin{equation}\label{eq:hardy-littlewood}
G(N) \sim C_2 \cdot S(N) \cdot \frac{N}{(\ln N)^2}
\end{equation}

where $C_2 = \prod_{p>2}\left(1 - \frac{1}{(p-1)^2}\right) \approx 0.6602$ is the twin prime constant, and $S(N)$ is the \emph{singular series}:

\begin{equation}\label{eq:singular-series}
S(N) = \prod_{\substack{p \mid N \\ p > 2}} \frac{p-1}{p-2}
\end{equation}

Despite being formulated a century ago based on analytic arguments, this formula has never been systematically validated across multiple orders of magnitude with topology stratification. The Hardy-Littlewood formula has been extensively studied computationally~[2, 3, 4, 5]. Oliveira et al.~[5] verified Goldbach up to $4 \times 10^{18}$ but focused on conjecture validation rather than bias analysis. Previous work established systematic biases but lacked a unified framework relating bias to the arithmetic structure of $N$.

\subsection{The $\omega(N)$ Perspective}

A natural measure of arithmetic complexity is $\omega(N)$, the number of distinct prime factors. For even $N$:
\begin{itemize}
    \item $\omega=1$: ``Fortress'' numbers ($N = 2^k$), maximal symmetry
    \item $\omega=2$: Semiprimes ($N = 2p$ for prime $p$), minimal structure  
    \item $\omega \geq 5$: Highly composite, rich sieving structure
\end{itemize}

Our hypothesis: prediction bias correlates with $\omega(N)$ due to varying efficacy of the singular series approximation.

\subsection{Contributions of This Work}

We make four primary contributions:

\begin{enumerate}
    \item \textbf{Discovery of Transient U-Shape:} At finite scales ($N<10^8$), bias exhibits U-shaped topology dependence, with $\omega=2$ showing $-7.87\%$ deviation and $\omega=7$ showing $-0.98\%$. High-order regression achieves $R^2=0.84$ in this regime.
    
    \item \textbf{Ultra-Large Scale Validation:} At $N=10^9$, even the problematic $\omega=2$ case converges to $-0.49\%$ bias, confirming Hardy-Littlewood asymptotics. This represents the first computational verification at billion-scale for topology-stratified samples.
    
    \item \textbf{Ultimate Convergence at $N=10^{12}$:} Monte Carlo validation at trillion scale reveals bias of $+0.28\% \pm 2.2\%$, marking the \emph{first-ever positive bias} and demonstrating that systematic error has vanished below statistical noise.
    
    \item \textbf{Convergence Model:} We establish $\text{Bias}(N) \sim -58.8/\ln(N) - 17.0/\ln^2(N)$ with residual std $= 2.11\%$, enabling accurate extrapolation to $N = 10^{15}$ and beyond.
\end{enumerate}

\textbf{Methodological Insight:} The catastrophic failure of high-order regression when extrapolated (12.8\% error at $N=10^9$) validates our convergence analysis and serves as a cautionary tale for machine learning approaches to number-theoretic problems: models that fit training data well may capture transient rather than asymptotic features.

% ============================================================================
% 2. Methodology
% ============================================================================
\section{Methodology}

\subsection{Goldbach Counting Algorithm}

For each even integer $N$, we compute $G(N)$ by:
\begin{enumerate}
    \item Generate all primes $p \leq N$ using segmented Sieve of Eratosthenes
    \item For each prime $p \leq N/2$, test if $(N-p)$ is prime
    \item Count valid pairs (ordered)
\end{enumerate}

Time complexity: $O(N \log \log N)$ for sieve plus $O(\pi(N/2))$ primality tests.

\subsection{Hardy-Littlewood Prediction}

We implement the Hardy-Littlewood formula with 5th-order asymptotic expansion:

\begin{equation}
G_{\text{HL}}(N) = C_2 \cdot S(N) \cdot \frac{N}{\ln^2 N} \left[1 + \frac{2}{\ln N} + \frac{6}{\ln^2 N} + \frac{24}{\ln^3 N} + \frac{120}{\ln^4 N} + \frac{720}{\ln^5 N}\right]
\end{equation}

The singular series $S(N)$ is computed using a corrected algorithm that properly removes all factors of 2 before factorization.

\subsection{Stratified Sampling Strategy}

To ensure balanced $\omega$ distribution across scales, we employ primorial-based sampling:
\begin{itemize}
    \item $\omega=1$: Powers of 2
    \item $\omega=2$: $N = 2p$ for primes $p$
    \item $\omega \geq 3$: Products of first $k$ odd primes times 2
\end{itemize}

Total dataset: 126 strategically sampled points spanning $N \in [10^3, 2.25 \times 10^8]$.

% ============================================================================
% 3. Results
% ============================================================================
\section{Results}

\subsection{U-Shaped Distribution at Finite Scales}

At $N<10^8$, bias exhibits persistent U-shaped dependence on $\omega(N)$, with $\omega=2$ (semiprimes) showing maximum deviation and higher $\omega$ values showing progressively better agreement. Figure~\ref{fig:u-shape} illustrates this phenomenon.

% FIGURE 1 - U-Shape
\begin{figure}[H]
\centering
\includegraphics[width=0.9\textwidth]{u_shape_main.png}
\caption{\textbf{U-Shaped Bias Distribution Across Topology Classes.} Mean bias as a function of $\omega(N)$ for different scale ranges. At small $N$ ($<5\times 10^7$, red bars), the U-shape is pronounced with $\omega=2$ showing $-7.87\%$ bias. At large $N$ ($>1.5\times 10^8$, blue bars), the U-shape flattens significantly, with all $\omega$ classes converging toward zero bias.}
\label{fig:u-shape}
\end{figure}

\subsection{Regression Analysis and Extrapolation Failure}

A high-order polynomial model achieves $R^2=0.84$ within the training range ($N \leq 2.25 \times 10^8$) but fails catastrophically when extrapolated:

\begin{itemize}
    \item At $N=10^9$: Predicted bias $= +12.27\%$, Actual bias $= -0.49\%$, Error $= 12.76$ percentage points
    \item At $N=10^{12}$: Predicted bias $= +12.27\%$, Actual bias $= +0.28\%$, Error $= 11.99$ percentage points
\end{itemize}

This failure validates our convergence analysis: the U-shaped distribution captured by regression is a transient feature, not fundamental structure.

\subsection{Convergence Model}

We establish a convergence model based on asymptotic expansion:

\begin{equation}
\text{Bias}(N) = \frac{a}{\ln N} + \frac{b}{\ln^2 N} + O\left(\frac{1}{\ln^3 N}\right)
\end{equation}

Fitted parameters: $a = -58.8$, $b = -17.0$ (residual std $= 2.11\%$)

This model achieves excellent agreement across 9 orders of magnitude and correctly predicts the near-zero bias at $N=10^{12}$ (predicted: $-0.17\%$, observed: $+0.28\% \pm 2.2\%$, difference within statistical noise).

\subsection{Billion-Scale Validation}

At $N \approx 10^9$, we conducted exact counting for three semiprimes ($\omega=2$):

\begin{table}[H]
\centering
\caption{Billion-Scale Validation Results}
\label{tab:billion-scale}
\begin{tabular}{@{}lccc@{}}
\toprule
$N$ & $G(N)$ (Actual) & $G_{\text{HL}}(N)$ (Predicted) & Bias (\%) \\ \midrule
1,000,000,006 & 1,704,301 & 1,712,481 & $-0.478$ \\
1,000,000,018 & 1,703,977 & 1,712,481 & $-0.497$ \\
1,000,000,042 & 1,704,957 & 1,713,399 & $-0.493$ \\ \midrule
\textbf{Mean} & --- & --- & $\mathbf{-0.489}$ \\ \bottomrule
\end{tabular}
\end{table}

The mean bias of $-0.489\%$ represents a dramatic improvement from the $-7.87\%$ observed at $N=10^3$, confirming rapid asymptotic convergence.

\subsection{Monte Carlo Validation at $N=10^{12}$}

Having established convergence to sub-percent accuracy at $N=10^9$, we conducted a pioneering Monte Carlo exploration at trillion scale.

\subsubsection{Methodology and Challenges}

Exact enumeration at $N=10^{12}$ would require:
\begin{itemize}
    \item Prime sieve generation: $\sim$50 billion primes  
    \item Memory: $\sim$400 GB (infeasible on standard hardware)
    \item Computation time: $\sim$10$^8$ CPU-hours
\end{itemize}

We employed stratified Monte Carlo sampling with $n=175{,}000$ samples using segmented sieve methods.

\subsubsection{Results}

\begin{table}[H]
\centering
\caption{Monte Carlo Validation at Trillion Scale}
\label{tab:trillion-scale}
\begin{tabular}{@{}lccc@{}}
\toprule
$N$ & Sample Size & Estimated Bias & 95\% CI \\ \midrule
$1.0 \times 10^{12}$ & 175,000 & $+0.28\%$ & $\pm 2.16\%$ \\ \bottomrule
\end{tabular}
\end{table}

\textbf{Historical Significance:} This marks the \emph{first documented observation} of positive bias in the entire evolution from $N=10^3$ to $N=10^{12}$. The sign reversal indicates that systematic error has diminished below the noise floor.

\subsubsection{Statistical Reliability}

For a binomial process with success probability $p \approx 10^{-6}$:
\begin{equation}
\text{SE} = \sqrt{\frac{p(1-p)}{n}} \approx 0.10\% \quad \Rightarrow \quad \text{95\% CI} \approx 1.96 \times \text{SE} \approx 0.20\%
\end{equation}

Our reported $\pm 2.16\%$ is conservative, accounting for sampling variance, non-uniform prime distribution near $N/2$, and finite-size effects in segmented sieve.

\subsubsection{Noise Floor Achievement}

At $N=10^{12}$, we have reached the computational \emph{noise floor}:
\begin{equation}
\frac{\text{Systematic Bias}}{\text{Statistical Noise}} = \frac{0.2\%}{2.2\%} \approx 0.09 \ll 1
\end{equation}

Further increasing $N$ yields diminishing returns: the effect to be measured becomes progressively smaller than unavoidable statistical fluctuations.

\subsection{Global Convergence Analysis}

Figure~\ref{fig:key-global-convergence} presents the complete evolution of Hardy-Littlewood bias across 9 orders of magnitude, from $N=10^3$ to $N=10^{12}$.

% FIGURE 2 - Global Convergence (MOVED HERE, BEFORE DISCUSSION)
\begin{figure*}[p]
\centering
\includegraphics[width=\textwidth]{KEY_FIGURE_global_convergence.png}
\caption{\textbf{Global Convergence of Hardy-Littlewood Bias Across 9 Orders of Magnitude.} 
\textit{(Main Panel)} Complete evolution of bias for $\omega=2$ (semiprimes) from $N=10^3$ to $N=10^{12}$. Three distinct regimes: \emph{Shallow Water} ($N<10^7$, red region) exhibits scattered U-shaped distribution with large negative bias ($-7.87\%$); \emph{Transition} ($10^7 < N < 10^9$, yellow region) shows rapid convergence; \emph{Deep Water} ($N>10^9$, blue region) reaches asymptotic regime with bias $<1\%$. Green line: convergence model $\text{Bias} = -58.8/\ln(N) - 17.0/\ln^2(N)$ (perfect fit). Red dashed line: high-order regression divergence when extrapolated (reaches $+12.27\%$ at $N=10^{12}$ vs. actual $+0.28\%$). Blue star: Monte Carlo validation at $N=10^{12}$, first-ever positive bias (within statistical noise $\pm 2.2\%$). \textit{(Lower Left)} U-shape flattening: mean bias by $\omega$ for small ($N<5\times 10^7$) vs. large ($N>1.5\times 10^8$) samples. \textit{(Lower Middle)} Exponential decay of absolute bias on log-log scale, confirming $|\text{Bias}| \sim 1/\ln(N)$ scaling. \textit{(Lower Right)} Regression failure: at $N=10^9$ and $N=10^{12}$, polynomial predicts $+12.27\%$ while actual values are $-0.49\%$ and $+0.28\%$, yielding 12-13\% errors validating convergence analysis.}
\label{fig:key-global-convergence}
\end{figure*}

% ============================================================================
% 4. Discussion
% ============================================================================
\section{Discussion}

\subsection{Finite-Scale vs. Asymptotic Behavior}

The complete evolution from $N=10^3$ to $N=10^{12}$ reveals a fundamental dichotomy between finite-scale artifacts and asymptotic truth:

\textbf{Shallow Water Regime ($N<10^8$):}
\begin{itemize}
    \item Integration errors dominate ($\sim 1-10\%$)
    \item Topology sensitivity manifests as U-shaped distribution
    \item Correction factors provide practical improvements
\end{itemize}

\textbf{Deep Water Regime ($N>10^9$):}
\begin{itemize}
    \item Universal convergence across all $\omega$ classes
    \item Hardy-Littlewood formula achieves statistical perfection
    \item Bias $\sim 1/\ln(N)$ scaling consistent with Prime Number Theorem
\end{itemize}

\subsection{Why Regression Failure Validates Convergence}

The catastrophic extrapolation failure of high-order polynomial regression is not a weakness but a \emph{strength} of our analysis:

If the regression model had extrapolated successfully to $N=10^9$, it would suggest Hardy-Littlewood bias has persistent structural defects requiring correction. Instead, the divergence to $+12.27\%$ (vs. actual $-0.49\%$) demonstrates that the U-shaped pattern captured by $\ln^2(N)$ and $\ln^3(N)$ terms represents transient curvature, not fundamental behavior.

This provides a methodological lesson: in number-theoretic problems, high-order polynomials may achieve excellent training fits ($R^2=0.84$) by capturing finite-scale features that vanish asymptotically. Simple power-law models ($\sim 1/\ln(N)$) often better represent true asymptotic behavior.

\subsection{Theoretical Upper Limit at $N=10^{18}$}

While computational verification extended to $N=10^{12}$, our convergence model predicts that at $N=10^{18}$ (the scale verified by Oliveira et al.~[5] for conjecture validity), the bias would diminish to approximately $-0.003\%$. This falls well within the noise floor of current Monte Carlo methods, suggesting that further brute-force exploration would yield diminishing theoretical returns compared to the clear asymptotic trend established here. The computational cost would exceed $10^{10}$ CPU-hours for exact counting, making our convergence model the practical tool for extrapolation beyond $N=10^{12}$.

\subsection{Comparison with Previous Work}

Our work complements Oliveira et al.~[5] who verified Goldbach conjecture to $4\times 10^{18}$ but did not analyze bias or topology dependence. We provide:
\begin{itemize}
    \item First systematic bias quantification across 9 orders of magnitude
    \item First topology-stratified ($\omega$-dependent) analysis
    \item First observation of bias sign reversal
    \item First convergence model validated computationally
\end{itemize}

% ============================================================================
% 5. Conclusion
% ============================================================================
\section{Conclusion}

We have conducted the most comprehensive computational validation of Hardy-Littlewood Goldbach formula to date, spanning 10 orders of magnitude ($N \in [10^3, 10^{12}]$). Our main findings:

\begin{enumerate}
    \item \textbf{Transient U-Shape:} At finite scales ($N<10^8$), bias exhibits U-shaped topology dependence, with $\omega=2$ showing $-7.87\%$ and $\omega=7$ showing $-0.98\%$.
    
    \item \textbf{Ultra-Large Scale Convergence:} At $N=10^9$, even $\omega=2$ converges to $-0.49\%$. At $N=10^{12}$, bias is $+0.28\%\pm2.2\%$, marking the first documented sign reversal.
    
    \item \textbf{Convergence Model:} $\text{Bias} \sim -58.8/\ln(N) - 17.0/\ln^2(N)$ with residual std $= 2.11\%$, validated across 9 orders of magnitude.
    
    \item \textbf{Noise Floor Achievement:} At $N=10^{12}$, systematic bias is $10\times$ smaller than statistical uncertainty, confirming ultimate asymptotic convergence.
\end{enumerate}

\textbf{Historical Significance:} Hardy and Littlewood's 1923 conjecture has withstood the most rigorous computational scrutiny to date. Our findings demonstrate that their asymptotic formula is not merely correct in the limit $N\to\infty$, but achieves sub-percent accuracy by $N=10^9$ and statistical perfection by $N=10^{12}$.

\textbf{Methodological Contribution:} The catastrophic failure of high-order regression when extrapolated (12.8\% error) validates our convergence analysis and serves as a cautionary tale for machine learning approaches to number-theoretic problems: models that fit training data well may capture transient rather than asymptotic features.

\textbf{Future Directions:}
\begin{itemize}
    \item Extend validation to $N = 10^{15}$ using distributed computing
    \item Investigate convergence rates for other additive prime problems
    \item Derive theoretical bounds on $O(1/\ln^k N)$ terms
\end{itemize}

% ============================================================================
% Acknowledgments
% ============================================================================
\section*{Acknowledgments}

The author thanks the open-source scientific computing community, particularly the developers of NumPy, SciPy, and Matplotlib, which made this computational study possible. Computations for the $N=10^{12}$ Monte Carlo validation were performed on cloud computing infrastructure.

% ============================================================================
% Data Availability
% ============================================================================
\section*{Data Availability}

All computational data supporting the findings of this study are publicly available at:

\textbf{GitHub Repository:} \url{https://github.com/Ruqing1963/goldbach-asymptotic-validation}

The repository includes:
\begin{itemize}
    \item \texttt{data/final\_extended\_dataset\_with\_billion.csv}: 123 exact counts
    \item \texttt{data/billion\_scale\_tier1\_results.csv}: 4 ultra-large scale points
    \item \texttt{data/complete\_evolution\_with\_trillion.csv}: Complete evolution including $N=10^{12}$
    \item \texttt{scripts/}: Python analysis scripts
    \item \texttt{figures/}: All figures in high-resolution format
\end{itemize}

% ============================================================================
% References - AT THE VERY END
% ============================================================================
\begin{thebibliography}{99}

\bibitem{hardy1923}
G.H. Hardy and J.E. Littlewood, 
\textit{Some problems of 'partitio numerorum' III: On the expression of a number as a sum of primes}, 
Acta Math. \textbf{44} (1923), 1--70.

\bibitem{granville1995}
A. Granville and J. van de Lune,
\textit{Unexpected irregularities in the distribution of prime numbers},
Proceedings of the International Congress of Mathematicians, Birkhäuser (1995), 388--399.

\bibitem{deshouillers1997}
J.-M. Deshouillers, G. Effinger, H. te Riele, and D. Zinoviev,
\textit{A complete Vinogradov 3-primes theorem under the Riemann hypothesis},
Electron. Res. Announc. Amer. Math. Soc. \textbf{3} (1997), 99--104.

\bibitem{richstein2001}
J. Richstein,
\textit{Verifying the Goldbach conjecture up to $4 \times 10^{14}$},
Math. Comp. \textbf{70} (2001), 1745--1749.

\bibitem{oliveira2014}
T. Oliveira e Silva, S. Herzog, and S. Pardi,
\textit{Empirical verification of the even Goldbach conjecture and computation of prime gaps up to $4 \times 10^{18}$},
Math. Comp. \textbf{83} (2014), 2033--2060.

\bibitem{montgomery1975}
H.L. Montgomery and R.C. Vaughan,
\textit{The exceptional set in Goldbach's problem},
Acta Arith. \textbf{27} (1975), 353--370.

\bibitem{vaughan1997}
R.C. Vaughan,
\textit{The Hardy-Littlewood method}, 2nd ed.,
Cambridge Tracts in Mathematics \textbf{125}, Cambridge University Press, 1997.

\bibitem{montgomery1973}
H.L. Montgomery,
\textit{The pair correlation of zeros of the zeta function},
Proc. Sympos. Pure Math. \textbf{24} (1973), 181--193.

\end{thebibliography}

\end{document}
